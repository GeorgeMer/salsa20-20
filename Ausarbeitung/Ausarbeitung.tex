% Diese Zeile bitte -nicht- aendern.
\documentclass[course=erap]{aspdoc}
\usepackage{amssymb}

%%%%%%%%%%%%%%%%%%%%%%%%%%%%%%%%%
%% TODO: Ersetzen Sie in den folgenden Zeilen die entsprechenden -Texte-
%% mit den richtigen Werten.
\newcommand{\theGroup}{215} % Beispiel: 42
\newcommand{\theNumber}{A500} % Beispiel: A123
\author{David Csida \and Georgios Merezas \and Fabian Degen}
\date{Sommersemester 2022} % Beispiel: Wintersemester 2019/20
%%%%%%%%%%%%%%%%%%%%%%%%%%%%%%%%%

% Diese Zeile bitte -nicht- aendern.
\title{Gruppe \theGroup{} -- Abgabe zu Aufgabe \theNumber}

\begin{document}
\maketitle

\section{Einleitung}
\subsection{Einführung}
Sei es beim Herstellen einer Verbindung mit einem Server oder das chatten mit Freunden und Bekannten rund um den Globus, nie genoss die Kryptographie mehr Relevanz als jetzt.
Man möchte nicht, dass die Nachrichten die man an seine Geliebten versendet von einer dritten Person mitgelesen werden können. Darum existieren kryptographische Verfahren zur Verschlüsselung von Informationen.
Eines dieser Verschlüsselungsverfahren ist Salsa20/20, welches es zu implementieren galt.

\subsection{Definitionen}
Im Folgenden bezeichne: 
\\ \hspace*{5mm} 1. $x \lll y$ die Linksrotation von $x$ um $y$ Bit. 
\\ \hspace*{5mm} 2. $\oplus$ den Binären XOR-Operator.

\subsection{Funktionsweise Salsa20/20}
Salsa 20/20 ist eine Stromchiffre basierend auf einem sogenannten Add-Rotate-XOR-Schema, ins Leben gerufen von David J. Bernstein.
\\Salsa20/20 verwendet einen Kern, eine 4x4 Matrix bestehend aus vorzeichenlosen 32-bit Little-Endian-Ganzzahlen, welcher durch Add, Rotate und XOR Operationen auf bestimmen Startwerten erzeugt wird.
Auf diesem generierten Kern wird dann mit der zu verschlüsselnden Nachricht Byte für Byte eine XOR-Operation ausgeführt.

\subsubsection{Der Salsa20-Kern:}
Der Salsa20-Kern ist eine Funktion, welche einen 64-Byte Block generiert, mit dem wie oben beschrieben auf der zu verschlüsselnden Nachricht Byte für Byte eine XOR-Operation augeführt wird.
Der 64-Byte-Block wird in 20 sogenannten Runden aus einem 256-Bit-Key, einer 64-Bit Nonce und einem 64-Bit Counter generiert.
Den Startzustand bildet dabei folgende 4x4 Matrix:
\[
    \begin{pmatrix}
    0\text{x}61707865 & K_0 & K_1 & K_2\\
    K_3 & 0\text{x}3320646\text{e} & N_0 & N_1\\
    C_0 & C_1 & 0\text{x}79622\text{d}32 & K_4\\
    K_5 & K_6 & K_7 & 0\text{x}6\text{b}206574
    \end{pmatrix}
\]



\section{Lösungsansatz}


\section{Korrektheit}


\section{Performanzanalyse}


\section{Zusammenfassung und Ausblick}

% TODO: Fuegen Sie Ihre Quellen der Datei Ausarbeitung.bib hinzu
% Referenzieren Sie diese dann mit \cite{}.
% Beispiel: CR2 ist ein Register der x86-Architektur~\cite{intel2017man}.
\bibliographystyle{plain}
\bibliography{Ausarbeitung}{}

\end{document}
